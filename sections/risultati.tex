\section{Risultati}
\label{sec:risultati}

In questa sezione verranno presentati i risultati degli esperimenti sulle varie euristiche. Gli esperimenti sono stati eseguiti su un totale di 40 problemi diversi. 

Per valutare le euristiche, sono state utilizzate diverse metriche:

\begin{itemize}
\item \textbf{Media passi}: la media del numero di passi necessari per chiudere il problema, calcolata su tutti i problemi;
\item \textbf{Media fissati}: la media di nodi fissati ad ogni iterazione, calcolata su tutte le iterazioni di tutti i problemi;
\item \textbf{Media max nodi fissati}: per ogni problema, viene calcolato il numero di nodi fissati ad ogni iterazione. Viene preso il numero di nodi fissati dell'iterazione che ne ha fissati di più per ogni problema, e viene effettuata la media di tale valore su ogni problema;
\item \textbf{Media nodi no fissati:} la media del numero di iterazioni in cui non vengono fissati nodi, calcolata considerando tutte le iterazioni di tutti i problemi;
\item \textbf{Max passi}: numero di passi necessari per chiudere il problema che ha richiesto più iterazioni;
\item \textbf{Max fissati}: numero massimo di nodi fissati in un passo, calcolato considerando ogni iterazione di ogni problema;
\item \textbf{Max max nodi fissati}:
\item \textbf{Min nodi no fissati}: minimo numero di iterazioni in cui non sono stati fissati nodi, calcolato su tutte le iterazioni di tutti i problemi.
\end{itemize}

La tabella mostra i risultati delle euristiche proposte sulle metriche presentate sopra. I migliori risultati per ogni metrica sono evidenziati in grassetto. 

Come si può vedere, l'euristica \textit{noAuxnoThornMaxPeso}, insieme all'euristica \textit{maxPeso}, sono quelle che, generalmente, portano ai migliori risultati. Non solo: i risultati di entrambe le euristiche sono gli stessi su ogni metrica. Questo significa che l'euristica \textit{maxPeso}, che favorisce i nodi a cui è assegnato peso maggiore, sembra essere la migliore per i problemi considerati, ma che prendere in considerazione la classe del nodo oltre al suo peso non porta alcuna variazione nei risultati.


